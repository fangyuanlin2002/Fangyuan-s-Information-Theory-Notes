\documentclass[../main.tex]{subfiles}
\begin{document}
\chapter{Strong Typicality}
\section{Strong AEP}
The setup is as follows: 
\begin{itemize}
    \item $\{X_k\}$ is a sequence of i.i.d. random variables $\sim p(x)$.
    \item $X$ denotes generic r.v. with entropy $H(X)<\infty$
    \item $\vec X=(X_1,\dots,X_n)$. Then \[
    p(\vec X) = p(X_1)\dots p(X_n)
    \]
    \item New assumption: $|\X| < \infty$
    \item Let the base of the logarithm be $2$, so $H(X)$ is in bits.
\end{itemize}
Notations:\begin{itemize}
    \item Let $N(x;\vec x)$ ne the number of occurrences of $x$ in the sequence $\vec x$.
    \item $n^{-1}N(x;\vec x)$ is called the relative frequency of $x$ in $\vec x$.
    \item $\{\frac{1}{n}N(x;\vec x): x\in\X\}$ is called the empirical distribution of $\vec x$.
\end{itemize}
\begin{pbox}{Example of Empirical Distribution}
    Let $\vec x=(1,3,2,1,1)$. Then $N(1;\vec x)=3$ and $empircalDist(1)=\frac{3}{5}$.
\end{pbox}
\begin{pbox}{Strongly Typical Set}
    The stronly typical set $T^{n}_{X^{\delta}}$ with respect to $p(x)$ is the set of sequences $\vec x=(x_1,\dots,x_n)\in \X^n$ such that \[
    N(x;\vec x)=0
    \] for $x\notin S_X$ and \[
    \sum_x|\frac{1}{n}N(x;\vec x)-p(x)|\leq \delta
    \]
    \begin{remark*}
        \begin{itemize}
            \item If $\sum_x|\frac{1}{n}N(x;\vec x)-p(x)|$ is small, then each term $|\frac{1}{n}N(x;\vec x)-p(x)|$ in the summand is also small.
            \item If $\vec x$ is strongly typical, the empirical distribution of $\vec x$ is approximately equal to the true distribution $p(x)$.
        \end{itemize}
    \end{remark*}
\end{pbox}
\begin{bbox}{Strong AEP}
    There exists $\eta > 0$ such that $\eta \to 0$ as $\delta\to 0$ and the following is true:
    \begin{itemize}
        \item If $\vec x$ is strongly $\delta$-typical, then \[2^{-n(H(X))+\eta} \leq p(\vec x) \leq 2^{-nH(X)-\eta}
        \]
        \item For $n$ sufficiently large, \[
        Pr(\vec X \in T_\delta) > 1-\delta
        \]
        \item For $n$ sufficiently large, \[
        (1-\delta)2^{n(H(\vec X)-\eta)}\leq |T^{n}_\delta| \leq 2^{n(H(\vec X)+\eta)}
        \]
    \end{itemize}
    The form of the strong AEP is very similar to the form of the weak AEP.
\end{bbox}
\begin{proof}
Idea:
    If $\vec x$ is strongly typical, then the empirical distribution is "good" (close to the real distribution). If the empirical distribution is "good," the sample entropy should be close to the real entropy, i.e. \[
    -\frac{1}{n}\log p(\vec x) \approx H(X),
    \] meaning $p(\vec x)\approx 2^{-nH(X)}$.
    \newline
    For any $\vec x \in T^n_\delta$ that is $\delta$ typical, we have \[
    p(\vec x)=p(x_1)\dots p(x_n) = \prod_{x\in \S_X}p(x)^{N(x;\vec x)} > 0.
    \] Then consider \begin{align*}
        \log p(\vec x) &= \sum_x N(x;\vec x)\log p(x)\\
        &= \sum_x (N(x;\vec x)-np(x)+np(x))\log p(x)\\
        &=n\sum_x p(x)\log p(x) - n\sum_x(\frac{1}{n}N(x)-p(x))(-\log p(x))\\
        &=-n [H(X)+\sum_x(\frac{1}{n}N(x)-p(x))(-\log p(x))].
    \end{align*}
    Since $\vec x\in T$, \[
    \sum_x|\frac{1}{n}N(x;\vec x)-p(x)| \leq \delta.
    \]
    Now consider the triangle inequality,
    \begin{align*}
        &|\sum_{x}(\frac{1}{n}N(x;\vec x)-p(x))(-\log p(x))|\\
        &\leq \sum_{x}|\frac{1}{n}N(x;
        \vec x)-p(x)|(-\log p(x))\\
        &\leq - \log(\min_x p(x))\sum_{x}|\frac{1}{n}N(x;
        \vec x)-p(x)|\\
        &\leq -\delta \log(\min_x p(x))\\
        &=\eta
    \end{align*} where \[
    \eta = -\delta\log(\min_xp(x)).
    \]
    Therefore, \[
    -\eta\leq \sum_x(\frac{1}{n}N(x;\vec x)-p(x))(-\log p(x))\leq \eta.
    \]
    Then we get the desired inequality by taking exponential.\\
    Note that $\eta$ tends to $0$ as $\delta$ tends to $0$.\\

    2): By WLLN, with probability tending to $1$, the empirical distribution of $\vec X$ is close to $p(x)$, so by definition $\vec X$ is strongly typical. Consider \begin{align*}
        N(x;\vec x)=\sum_{k=1}^{n}I_k(x)
    \end{align*} where $I_k$ is an indicator for the event $X_k=x$. Then $I_k(x)$ are iid and $E(B_k(x))=p(x). $.\\
    By WLLN, for $n$ sufficiently large, $Pr(|\frac{1}{n}\sum_{k=1}^{n}I_k(x)-p(x)|>\frac{\delta}{|\X|}) < \frac{\delta}{|\X|}$.\\
    Then consider the probability that for some $x\in \X$, the empirical probability deviates from the true probability, we use the union bound to show is the this probability is upper bounded by $\delta$.\\
    Note that $\sum_x|\frac{1}{n}N(x;\vec x)-p(x)|>\delta$ implies $|\frac{1}{n}N(x;\vec x)-p(x)|>\frac{\delta}{||\X}$ for some $x$ (at least one of them is greater than the average). We have the desired inequality.
    \newline
    3): $ |T| lower_bound(p(\vec x))\leq P(T) \leq 1$
\end{proof}
\begin{bbox}{6.3 Probability that a sequence is not strongly typical tends to $0$ exponentially fast}
    For sufficiently large $n$, there exists $\phi(\delta)>0$ such that \[
    P(\vec X\notin T^{n}_\delta)) < 2^{-n\phi(\delta)}.
    \]
The proof uses the Chernoff bound.
\end{bbox}

\section*{Strong Typicality versus Weak Typicality}
\begin{remark*}
    \begin{itemize}
        \item Weak Typicality: empirical entropy is close to the true entropy
        \item Strong Typicality: empirical distribution is close to the true distribution
        \item Strong typicality $\implies$ weak typicality.
        \item Weak typicality $\not\Rightarrow$ strong typicality
        \item Strong typicality works only for finite alphabet, i.e. $|\X|<\infty$, but weak typicality works for any countable alphabet.
    \end{itemize}
\end{remark*}
\begin{bbox}{Strong typicality implies weak typicality}
If $\vec x\in T^{n}_\delta$, then $x\in W^{n}_{\eta}$ where $\eta \to 0$ as $\delta \to 0$.
\begin{proof}
    Let $\vec x$ be strongly $\delta$-typicality, then $2^{-n(H(X)+\eta)}\leq p(\vec x)\leq 2^{-n(H(X)-\eta)}$. Which means $H(X)-\eta \leq -\frac{1}{n}\log p(\vec x)\leq H(X)+\eta$.
\end{proof}
\end{bbox}
\begin{pbox}{Weak Typicality does not imply strong typicality}
    Consider $X$ with distribution $p$ such that \[
    p(0)=0.5, p(1)=0.25, p(2)=0.25.
    \] Consider a sequence $\vec x$ of length $n$ and let $q(x)=n^{-1}N(x;\vec x)$ be the relative frequency of occurrence of symbol $x$ in $\vec x$, where $x=0,1,2$.
    Consider \begin{align*}
        -\frac{1}{n}\log p(\vec x) &= -\frac{1}{n}\sum_k \log p(x_k)\\
        &= -\frac{1}{n}[N(0;\vec x)\log p(0) + N(1;\vec x)\log p(1)+N(2;\vec x)\log p(2)]\\
        &=q(0)\log p(0)+q(1)\log p(1)+q(2)\log p(2)        
    \end{align*}
    While real entropy is \[
    p(0)\log p(0)+p(1)\log p(1)+p(2)\log p(2)
    \]
    If we have half $0$ and half $1$, then the empirical entropy is exactly the same as the true entropy.
    \end{pbox}

    
\section{Joint Typicality}
Setup:
\begin{itemize}
    \item $\{(X_k, Y_k)\}$ where $(X_k, Y_k)$ are iid $\sim p(x,y)$
    \item $(X,Y)$ denotes pair of generic r.v. with entropy $H(X,Y)<\infty$
    \item $|\X|,|\Y| < \infty$
\end{itemize}
Notations
\begin{itemize}
    \item Consider $(x,y)\in \X^n\times \Y^n$, let $N(x,y;\vec x,\vec y)$ be the number of occurrences of $(x,y)$ in the pair of sequences $(\vec x, \vec y)$
    \item $\frac{1}{n}N(x,y;\vec x,\vec y)$ would be the relative frequency of the tuple $(x,y)$
\end{itemize}
\begin{pbox}{Strongly jointly typical set}
The strongly jointly typical set $T^n_{\delta}$ with respect to $p(x,y)$ is the set of sequence pairs $(\vec x, \vec y)$ such that \[
\sum_x \sum_y|\frac{1}{n}N(x,y;\vec x,\vec y)-p(x,y)|\leq \delta
\]    
\end{pbox}
\begin{bbox}{Consistency}
    If $(\vec x, \vec y)$ is jointly typical, then $\vec x$ and $\vec y$ are both marginally typical.
\end{bbox}
\begin{bbox}{Preservation}
    Let $Y=f(X)$. If \[
    \vec x =(x_1,\dots,x_n)\in T_X
    \], then $f(\vec x)=(y_1,\dots,y_n)\in T_Y$ where $y_i=f(x_i)$
\end{bbox}
\begin{bbox}{Strong JAEP}
    Let $(\vec X,\vec Y)$ be a pair of sequences of iid tuple, then there exists $\lambda>0$ such that $\lambda\to 0$ as $\delta\to 0$ and the following hold:
    \begin{itemize}
        \item If $(\vec x,\vec y)\in T$, then \[
        2^{-nH(X,Y)+\lambda}\leq p(\vec x,\vec y)\leq 2^{-n(H(X,Y)-\lambda}
        \] 
        \item For n sufficiently large, \[
        Pr((\vec X,\vec Y) \in T)>1-\delta
        \]
        \item For $n$ sufficiently large, \[
        (1-\delta)2^{n(H(X,Y)-\lambda)}\leq |T|\leq 2^{nH(X,Y)+\lambda}
        \]
    \end{itemize}
\end{bbox}
\begin{bbox}{Stirling's Approximation}
    \[
    \ln n! \approx n\ln n
    \]
    \begin{proof}
        \[
        \ln n!=\ln 1+\dots+\ln n
        \]
        Since $\ln x$ is monotonically increasing, we have \[
        \int_{k-1}^k
\ln x \dd x < \ln k <  \int_{k}^{k+1}
\ln x \dd x\]
    \end{proof}
Sum over $1\leq k\leq n$, we have \[
        \int_{0}^n
\ln x \dd x < \ln k <  \int_{1}^{n+1}
\ln x \dd x\]
ie \[
n\ln n-n< \ln n!<(n+1)\ln (n+1)-n
\]
\end{bbox}
\begin{bbox}{Binomial coefficient}
    \[
    \binom{n}{np,n(1-p)}\approx 2^{n H_2(p)}
    \]
    \begin{proof}
        \[
        \binom{n}{np,n(1-p)} = \frac{n!}{(np)!(n(1-p))!}
        \]
        Then take logarithm, we have \[
        \ln \binom{n}{np,n(1-p)} = \ln n!-\ln (np!-\ln (n(1-p))!
        \]
        This quentity is approximately $n \H_e(\{p,1-p\})$ by Sterling's approximation. Lastly, change the base to $2$.
    \end{proof}
\end{bbox}

\section{Conditional Strong AEP}
\begin{bbox}{Conditional Strong AEP}
For any $\vec x \in T$, define \[
T^n_{Y|X_\delta}(\vec x) = \{\vec y\in T^n_{Y}:(\vec x, \vec y)\in T^n_{XY}
\]
If $|T_{Y|X}|>1$, then \[
2^{n(H(Y|X)-\nu)} \leq |T^n_{Y|X}(\vec x)| \leq 2^{n(H(Y|X)+\nu)}
\]
where $\nu\to 0$ as $n\to \infty$ and $\delta\to 0$
\end{bbox}
Note that since \[
\frac{|T_{Y|X}|}{|T_X|}\approx \frac{2^{nH(X,Y)}}{2^{nH(X)}} = 2^{n(H(X,Y)-H(X))}=2^{nH(Y|X)}
\], so the number of $\vec y$ that are jointly typical with a typical $\vec x$ is approximately equal to $2^{nH(Y|X)}$

\begin{bbox}{Upper bound of conditional SAEP}
   If $|T^n_{Y|X_\delta}(\vec x)|\geq 1$, then \[
   |T^n_{Y|X_\delta}(\vec x)\leq 2^{n(H(Y|X)+\nu)}|
   \] where $\nu\to 0$ as $n\to \infty$ and $\delta\to 0$.
   \begin{proof}
       For any $\nu >0$, consider \begin{align*}
          2^{-n(H(X)-\nu/2)}&\geq p(\vec x) \quad \text{by typicality of $\vec x$}\\
          &=\sum_y p(x,y)\\
          &\geq \sum_{y\in T^n_{Y|X}}p(x,y)
           \\
          &\geq \sum_{y\in T_{Y|X}} 2^{-n(H(X,Y)+\nu/2)} \quad\text{by joint typicality}\\
          &=|T^n_{Y|X}| 2^{-n(H(X,Y)+\nu/2)}
       \end{align*}
   \end{proof}
\end{bbox}
The proof of the lower bound is more complicated.
\begin{pbox}{Lower Bound}
    Consider $\X=\{0,1\}$, $\{\Y=\{a,b,c\}$.
    There are probably $n p(0,a)$ occurrences of the pair $(0,a)$, etc. If we rearrange the components of $\vec y$, corresponding to $x_k=0$, then joint typicality is preserved. Then we can compute the number of arrangements. Then use the binomial coefficient approximation. As long as there's a typical $\vec y$, we can get a approx $2^{nH(Y|X)}$.
\end{pbox}
\begin{bbox}{6.12 Corollary}
    Let $S_X$ be the set of all sequences $\vec x\in T_X$ such that $T^n_{Y|X}$ is nonempty. Then \[
    |S_X|\geq (1-\delta)2^{n(H(X)-\psi)},
    \] where $\psi\to 0$ as $n\to \infty$ and $\delta\to 0$.
    This says $S_X$ and $T_X$ grows at the same asymptotic rate.
    \begin{proof}
        By consistentcy of strong typicality, if $(\vec x,\vec y)$ is jointly typical, $\vec x$ and $\vec y$ are marginally typical. Then \[T^n_{XY}=\bigcup_{\vec x\in S_X}\{(\vec x,\vec y):\vec y\in T^n_{Y|X}\}
        \]
        Using the lower bound on $|T^n_{XY}|$ in strong JAEP and upper bound on $T^n_{Y|X}$, we get the desired inequality, where $\psi$ is $\lambda + \delta$
    \end{proof}
    \begin{proposition*}
        With respect to a joint distribution $p(x,y)$, for any $\delta>0$,\[
        Pr\{\vec X\in S^{n}_X\} > 1-\delta
        \] for $n$ sufficiently large.
        \newline
        This says with high probability, we can obtain sequence such that there exists a $\vec y$ that is jointly typical with it. \end{proposition*}
\end{bbox}

Joint typicality has an "asymptotic quasi-uniform" structure. We can draw an array to visualize. The rows are $2^{nH(X)}$ typical $x$ sequences. The columns are $2^{nH(Y)}$ typical $y$ sequences. The total number of points in this array where $\vec x$ and $\vec y$ are jointly typical are $2^{nH(X,Y)}$. If we fix a typical $x$ sequence, the number of dots in that row is $2^{nH(Y|X)}$ dots. (every row in a strongly typical array has approx the same number of dots).
\centering{\textbf{Interpretation of the basic inequalities}}
Since the number of dots is less than or equal to the number of cells, we have that \[
2^{nH(X,Y)}\leq 2^{nH(X)}2^{nH(Y)}
\]
or $H(X,Y)\leq H(X)+H(Y)$\\
or \[
I(X,Y)\geq 0.
\]
The quasi-uniform array provides a combinatorial interpretation of information inequalities.
\end{document}