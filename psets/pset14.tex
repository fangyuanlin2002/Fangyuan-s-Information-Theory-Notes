\documentclass[../main.tex]{subfiles}
\begin{document}
\section*{Problem Set 14}
    Name: Fangyuan Lin, UC Berkeley, Class of 2024


\subsection*{11.1 Properties of Capacity $C(P)$}
\begin{enumerate}
    \item $C(P)=\frac{1}{2}\log(1+\frac{P}{N})$ is non-decreasing: \begin{proof}
        This is because $\log$ is non-decreasing. Also by allowing a greater $P$, the optimization problem is over a bigger set so the solution $C(P)$ would be greater.
        \end{proof}
    \item $C(P)$ is concave: \begin{proof}
        Look at the second derivative and see if it's non-positive.
        \[
        \frac{\dd C}{\dd P}=\frac{1}{2}\frac{1}{P+N}
        \] and \[
        C''(P)=-\frac{1}{2}\frac{1}{(P+N)} < 0.
        \]
    \end{proof}
    \item $C(P)$ is continuous: 
    \begin{proof}
        This follows from the continuity of $\log$.
    \end{proof}
\end{enumerate}
\subsection*{11.6 Calculation of the channel capacity}
Consider a system Gaussian of channels with noise vector $\vec Z\sim (\vec 0, K_{\vec Z})$ and input power constraint of $3$. Determine the capacity of the system for the following correlation matrices:
\begin{itemize}
    \item \[
    K_{\vec Z}=\begin{bmatrix}
        4 & 0 & 0\\
        0 & 5 & 0\\
        0 & 0 & 2
    \end{bmatrix}
    \]
    \begin{proof}
        \begin{enumerate}
            \item The water filling constant is $\nu$ such that \[
            (\nu - 4)^+ +(\nu-5)^+ +(\nu-2)^+ =3
            \] where the solution is \[
            \nu = 4.5.
            \]
            \item Then the capacity is \[
            C(3) = \frac{1}{2}\left[\log\left(1+\frac{4.5-4}{4}\right) + \log\left(1+\frac{4.5-2}{2}\right)\right]
            \]
        \end{enumerate}
    \end{proof}
    \item \[
    K_{\vec Z} = \begin{bmatrix}
        7/4 & \sqrt{2}/4 & -3/4\\
        \sqrt{2}/4 & 5/2 & -\sqrt{2}/4 \\
        -3/4 & -\sqrt{2}/4 & 7/4
    \end{bmatrix}
    \]
    \begin{proof}
        \begin{enumerate}
            \item This a uncorrelated system, but we can always decorrelate it while maintaining the same channel capacity.
            \item The eigenvalues of the given positive semidefinite matrix are $1,2,3$.
            \item Therefore, the water-filling constant \[
            \nu = 3
            \]
            \item Then apply the same formula as before to get \[
            C(3) = \frac{1}{2}\left[\log(1-\frac{3-1}{1}) + \log(1+\frac{3-2}{2})\right]
            \]
        \end{enumerate}
    \end{proof}
    \end{itemize}
\subsection*{11.7 Optimal input distribution for correlated Gaussian system}
In the system of correlated Gaussian channels, let $K_{\vec Z}$ be diagonalized as $Q\Lambda Q^T$ by spectral theorem. Let $A^*$ be the $k\times k$ diagonal matrix with $a_i^*$ on the diagonal, where $a_i^*$ is the optimal power allocated to the $i$th channel. Show that $\N(0,QA^*Q^T)$ is the optimal input distribution.
\begin{proof}
    \begin{enumerate}
        \item Goal: verify that the proposed input distribution achieves the channel capacity.
        \item We compute $K_{\vec Y}$: \begin{align*}
            K_{\vec Y} &= K_{\vec X}+K_{\vec Z} \quad \text{by independence of $\vec X$ and $\vec Z$}\\
            &= Q A^* Q^T + Q\Lambda Q^T \quad\text{by assumption}\\
            &= Q(A^*+\Lambda)Q^T \quad\text{Distribute}
        \end{align*}
        \item Now we compute the mutual information and we check that it is maximized: \begin{align*}
            \I(\vec X,\vec Y) &= h(\vec Y)-h(\vec Y|\vec X)\\
            &=h(\vec Y)-h(\vec Z|\vec X) \quad\text{by a previous lemma}\\
            &=h(\vec Y)-h(\vec Z)\\
            &= \frac{1}{2}\log(2\pi e\det K_{\vec Y})-\frac{1}{2}\log(2\pi e\det K_{\vec X})\\
            &=\frac{1}{2}\log(\det(Q(A^*+\Lambda)Q^TQ\Lambda^{-1}Q^T))\\
            &=\frac{1}{2}\log \det(A^*\Lambda^{-1}+I)\\
            &=\frac{1}{2}\sum_{i=1}^k\log(1+\frac{a_i^*}{\lambda_i}) \quad\text{here determinant is the product of diagonal entries}
        \end{align*}
    \end{enumerate}
\end{proof}
\end{document}